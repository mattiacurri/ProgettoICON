\section{Conclusioni}
\noindent

\noindent E' stato condotto uno studio sull'applicazione di tecniche di apprendimento supervisionato per classificare il \textit{Corporate Credit Rating} delle aziende. E' stata poi creata una rete bayesiana per fare inferenza su nuovi dati e per rispondere a query probabilistiche. I risultati ottenuti con le varie tecniche sperimentate sono state soddisfacenti, e nonostante la poca quantità di dati a disposizione, la rete bayesiana sembra rappresentare bene il dominio.

\section{Sviluppi futuri}
\noindent Oltre a quelli già proposti, si potrebbero considerare i seguenti sviluppi futuri:
\begin{itemize}[label=-]
    \item Provare a studiare le singole classi originali, avendo a disposizione più dati o con modelli come le reti neurali.
    \item Fare uno studio sul \textit{noise} dei rating, dovuto a fattori esterni da quelli utilizzati
    \item Fare \textit{feature engineering} con altre informazioni del dominio
    \item Fare \textit{feature selection} utilizzando tecniche più avanzate, e quindi non solo basandosi su correlazioni e conoscenza del dominio
    \item Sviluppo di un'interfaccia per poter utilizzare il modello in produzione 
\end{itemize}